%% LaTeX source of Chapter 6 of the thesis.
%% NEVER compile this file. Complie 'thesis.tex' instead.

\chapter{总结}
\label{Chapter 6}

从全文的整体结构来看,我们完成了高维时序数据压缩到可视化的一整套流程。首先我们介绍了时序数据的相关特征,时序数据压缩的相关历史研究,然后我们就时序
数据压缩所需要的工具与算法进行了介绍,在程序设计与实现部分我们给出了基于Java相关平台的设计与实现思路。在实验结果部分我们给出了压缩实例和压缩效果的
展示。

从压缩效果来看,本论文提供了一种切实可行的高维时序数据压缩的可选方案,从可视化的角度来看我们又提供了对于不同可视化工具的支持设计。我们的原则
是基于可扩展的接口设计,同时强调设计的可重复使用性。总体来说论文中的整体设计与实现是对高维时序数据与可视化的一次有益的探索。对于工程中的使用者来说我们的设计思路也有相当的参考价值。

\section{未来工作}
\label{Section 6.1}

针对高维时序数据压缩和可视化这一问题,以及我们的研究结果,
未来还有许多进一步的工作可以继续开展:
\begin{enumerate}[(1)]
	\item 在时序数据的压缩算法上我们应该还有很多需要改进的部分
	\item 时序数据压缩的计算量异常庞大,我们可以尝试基于分布式系统来提高我们的计算速度
	\item 如何构造一个更为准确的时序数据产生模型
	\item 对于压缩梯度的设置我们是否能够设计的更为准确
	\item 在程序设计部分离工程的标准还有一定的差距,我们应该以更为合理和优秀的设计为目标。
\end{enumerate}
