%% LaTeX source of Chapter 1 of the thesis.
%% NEVER compile this file. Complie 'thesis.tex' instead.

\chapter{引言}
\label{Chapter 1}

时序数据是按时间节点进行记录的数据集合,与非时序数据相比,它的顺序性要求更高,数据的应用范围更广。时序数据产生于生活中的各个方面,例如气
象观测中的大气数据,汽车发动机的各项传感器数据,这类数据不仅维度很高,同时产生的数据体量也非常大。目前随着时序数据规模的增大,不加处理的
原始数据已经对关系数据库特别是对一些移动平台和其它嵌入式设备造成了很大的存储压力。由于时序数据在一定时间内的重复率比较高,例如汽车的速度
在某段时间内都保持在一定速度,油料温度也都稳定在一定范围内,特别是对于发动机磨粒浓度这类数据很有可能长时间都稳定在某固定值,此类数据导致
了存储空间的严重浪费。高维时序数据是时序数据在空间上的扩展,相比于一维数据,多维时序数据复杂度更高,数据要求的存储空间更大。

理想的数据压缩是实现最小的存储成本,提高数据的传输效率,随着云计算和物联网的不断发展,数据的压缩存储技术成为海量数据研究的重要内容,根据
数据压缩后是否能完全还原为原始数据我们将数据压缩分为有损压缩和无损压缩,无损压缩的期望压缩比通常在1:2到1:5,相对于无损压缩,有损压缩
通常只要求压缩后的数据不影响特定范围内的正常使用,有损压缩广泛运用于视屏和音频的压缩。

单维时序数据已经存在很多的成熟的压缩算法,例如××××算法。相比于单维时序数据多维时序数据的压缩更为复杂,但是生活中需要对高维时序数据进行压
缩的场合却是很多,目前存在一些针对于时序数据的压缩算法,例如旋转门,稳态阈值,线性外插值等压缩算法,但是大多属于一维数据的有损压缩算法,
不能直接运用于多维时序数据的数据压缩,而且这些压缩算法的复杂度过高,海量的数据更是需要庞大的计算资源,而且这些压缩数据压缩后不能直接存储于关系数据库,这很大程度的影响了这些数据对于业务的使用,因此本论文设计了一种真对于高维时序数据的压缩方法,并提供了一种附加的持久化方式。

对于压缩处理过后的时序数据如何快速的从中找到自己需要的信息这是一个比较通用的业务场景,数据可视化的目的就是将这些晦涩难懂的数据在于用户的交互中展示出来。可视化的图表,更易于我们观察,对于发现数据的特征和深入的理解数据隐含的行为有更大的帮助,数据可视化不同于将所有的数据通过图标直接显示,数据体量超大的时序数据,直接用可视化系统观测不仅对可视化系统造成了巨大的压力,更重要的是这通常会直接导致高价值的数据被淹没。

根据现有的绝大部分业务场景,他们的持久化都是采用的基于关系操作的关系数据库例如PC端常用的MySQL,移动端广泛运用的SQLite,当然也不排除一些特殊的业务场景使用了类似于MongoDB的非关系数据库。在本论文中我们给出了一种基于蜡烛图的的海量时序数据的可视化方法,具体的可视化我们使用开源的图形化报表库,再在可视化的过程中我们我们先对时序数据进行高效的线性压缩,生成特定的区间,对于每个区间的统计数据,我们采用折线图和蜡烛图进行可视化。这种方法不仅能够对原始数据进行大范围的压缩,还能再压缩的基础之上通过用户的需求分层显示压缩后的效果

本论文依次介绍了相关的应用历史背景,算法实现,程序的相关实现等其它步骤,并在最后给出了一些针对于特定场合的压缩选项和优化。







