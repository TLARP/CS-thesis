%% LaTeX source of Chapter 2 of the thesis.
%% NEVER compile this file. Complie 'thesis.tex' instead.

\chapter{预备知识}
\label{Chapter 2}

\section{数据压缩简介}
\label{Section 2.1}


数据压缩的一般前提是不损失有用信息,理想的效果是达到最小的存储空间,同时提高传输和存储的效率。我们通过一定的算法重新组织数据,从而来减少
数据的冗余。数据压缩可分成两种类型,一种叫做无损压缩,另一种叫做有损压缩。无损压缩通常能保证在压缩数据后我们可以通过目标数据进行重构,
重构出的数据能和原始数据保持一致。常见的无损压缩包括磁盘文件的压缩,无损压缩的效果一般能把普通文件的数据压缩到原来的1:2到1:5.常用的
无损压缩算法包括霍夫曼算法,LZW(Lenpel-Ziv\&Welch)压缩算法。有损压缩是指在压缩后我们可以通过目标数据重构出原始数据的主要轮廓而这些
数据不会影响到我们对他们的使用。


\section{数据压缩的理论极限}
\label{Section 2.2}

了解了2.1中数据压缩的概念和相关分类,我们给出数据压缩的理论极限,值得一提的是对于时序数据的数据点压缩也存在压缩极限当数据压缩到一定程度之后,再提高压缩比例,带来的结果就是数据的永久性损害。

假设在均匀分布的情况下,我们采集到的某一维数据的取样值在存储文件中出现的概率是p,那么在这个取样值最多可能出现1/p种情况。因此我们需要log2(1/p)
个存储单位表示该采样数据。我们把这个结论可以推广到一般情况。假设我们采样的某一维时序数据由n个采样值组成,每个部分的内容在存储文件中的出现概率分别为
p1、p2、...pn。那么,替代符号占据的存储单位最少为下面这个表达式:
\begin{equation}
log2(1/p1) + log2(1/p2) + ... + log2(1/pn)= ∑ log2(1/pn)
\end{equation}
考虑单个采样值,它对应的存储单位为:
\begin{equation}
∑ log2(1/pn) / n= log2(1/p1)/n + log2(1/p2)/n + ... + log2(1/pn)/n
\end{equation}


\section{蜡烛图}
\label{Section 2.3}
K线图(candlestick charts)也叫蜡烛图,蜡烛图来源于对统计数据的分析,例如股价,其中包含股价的开盘价,收盘价,最高价和最低价。

K线图的状态可以分为翻转形态,整理形态等。K线图由于其细致明确的表达效果,引入到了股市和期货市场,K线图由统计区间的开始值,最大值,最小值,结束值组
成,假设给定一个统计区间,当我们确定区间的开始值和结束值时,我们把这两者中间的部分用对应的实体矩形画出,当开始值小于结束值时,我们用红色标注,此时k线被称为阳线,当开始值大于结束值时我们用黑色标注,此时对应的k线我们称之为阴线。

K线图绘制的技术指标和K线图的分析对于我们从中寻找有用的数据有很大的帮助。它可以作为我们从时序数据中寻找奇点数据最有力的工具,根据经典的K线图或者图上的某一指标分析和得出的结论不一定完全正确,但是结合我们时序数据压缩过后的数据查询,就能准确而快速的寻找到问题所在。

\textbf{注意:}时序数据绘制过程中的技术指标是对应于某一段时间内的统计分析数据,并不是和原始数据完全一样。因此这些数据的最终效果可能会受到人为因素
的控制,例如我们设置的压缩比过大,那么此时的绘图界面可能和原始数据相差较大。此时我们应该结合主要因素来判断图形是否能表达原始数据的真实趋势,以及我
们是否需要调整原始数据的压缩比和其它参数,以达到理想的效果。


\section{JAVA\&设计模式}
\label{Section 2.4}

高维时序数据压缩的研究成果适用于工程中的很多场景,而JAVA语言也是现阶段业务场合中使用最为频繁的编程语言,论文的设定环境为基于Java 
EE虚拟机的嵌入式平台,因此需要了解Java EE相关的一些基础知识。


JDBC是JAVA 语言中与数据库交互最为流行的方式,绝大部分数据库都有特定的JAR包来实现这些接口。而现在最流行的持久化框架包括Hibernate,ibatis都是基
于JDBC完成的,学会使用JDBC是完成该论文的重点。

设计模式(Design pattern)是一套反复被使用,多数人知晓,经过分类与编目的代码设计经验的总结,使用设计模式的目的是使代码的可用度达到最大化,同时使用良好的设计模式对于代码的维
护也有很重要的作用,在本论文中需要了解的设计模式包括线程安全的单例模式,工厂模式,原型模式,和装饰者模式已经访问者模式。








