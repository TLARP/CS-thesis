%% LaTeX source of Chapter 5 of the thesis.
%% NEVER compile this file. Complie 'thesis.tex' instead.

\chapter{程序设计与实现}
\label{Chapter 4}

\section{开发环境介绍}
\label{4.1}
\begin{enumerate}[(1)]
	\item 开发环境:eclipse
	\item 持久化:SQLite
	\item 项目构建工具:Maven
	\item 语言:Java\&JavaScript
\end{enumerate}

\section{需求分析}
\label{4.2}
\begin{enumerate}[(1)]
	\item 基于B/S 完成整个项目。
	\item 实现数据压缩算法,并绘制相应的可视化界面。
	\item 实现压缩选项参数的可定制化。
	\item 实现灵活可扩展的接口设计,能实现持久化平台和可视化工具的自由切换。
	\item 保证代码的可维护性,尽可能多的合理准确的增加对相应设计模式的使用。
	\item 通过代码实现,验证算法的可行性,并对算法给出改进。
\end{enumerate}

\section{UML类图}
\label{4.3}


\section{时序数据模拟设计}
\label{4.4}
\subsection{时序数据模拟器接口}
\begin{lstlisting}
public interface DataGen{
	public ArrayList<?> getGenData();
	public void run();
	public void stop();
}
\end{lstlisting}
注:DataGen接口含有一个模拟器的所有方法,一个模拟器的实现必须至少含有这些方法。

\subsection{时序数据模拟器工厂}
\begin{lstlisting}
public class DataGenFactory{
	public DataGen createDataGen();
}
\end{lstlisting}
注:DataGenFactory工厂提供出一个时序数据模拟器实例,调用者只需要查看DataGen的接口代码即可,不用关心DataGen的具体实现。


\subsection{持久化JDBC链接}
\begin{lstlisting}
public abstract class Instance{
	public void connection();
	public void startTransaction();
	public void insert();
	public void delete();
}
\end{lstlisting}
注:时序数据持久化类为抽象类,必须提供事务操作。

\subsection{时序数据模拟实现}
\begin{lstlisting}
public class SeriesDataGen implements DataGen{
	@override
	public ArrayList<?> getGenData();
	@override
	public void run();
	@override
	public void stop();

	private ArrayList arrayList;

	pivate createdate();//该方法产生对应的数据

	public static getDataGen(){//此处使用了一个单例模式
		if(dataGen==null)
			dataGen=new DataGen()
		
		return dataGen;
	}

	private static DataGen dataGen;
}
\end{lstlisting}
注:由于篇幅限制,我们给出了时序数据的产生接口设计,同时我们在上述过程中使用了工厂和单例的设计模式。




\section{数据压缩处理}
\label{4.5}

\subsection{时序数据压缩接口}
\begin{lstlisting}
public class DataCompress{
	private process(ArrayList<?> dataarray){

	}
	public void CompressInterval();//此方法为直接调用方法

	private void IntervalMerage();//区间合并算法

	private void getGenData();//数据生成算法

	public void react();//此方法为注册方法

}
\end{lstlisting}
注:此类包含了一个数据处理的单个过程。


\subsection{时序数据处理中心}
\begin{lstlisting}
public class processcore{
	private List<DataCompress> list;

	public void register(DataCompress unit);

	private listIterator;
}
\end{lstlisting}

注:此类包含了各个处理类的注册中心,当processCore接受到特定消息,包括stop,suspend等
消息,就可以调用相应的reac方法。这用到了迭代器和监听器模式。

\subsection{可视化的绘图}
可视化的绘图由JavaScript完成,可以参考附录对应的实现代码。
























